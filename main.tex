\documentclass[10pt]{beamer}

% Theme settings
\usetheme{CambridgeUS}
\usecolortheme{default}
\usefonttheme{professionalfonts}
\setbeamertemplate{navigation symbols}{}
\setbeamertemplate{footline}[frame number] % only show "1 / 9"


% Packages
% \usepackage{graphicx}
\usepackage{booktabs}
\usepackage{hyperref}
\usepackage{ragged2e}

% Metadata (short versions for footer)
\title[Seccomp Sandbox]{\textbf{Seccomp-BPF Sandbox: System Call Filtering for Container Security}}
\subtitle[OS (CS363)]{Operating Systems (Course Code: CS363)}
\author[Sawant, Singhal, Sodhi]{Group Members: \\ 202351028 \\ 202352301 \\ 202352314}
\institute[IIIT-V]{IIIT Vadodara \\ Gandhinagar Campus}
\date{October 2025}


\begin{document}

% ---------------------------
% Title Slide
% ---------------------------
\begin{frame}
  \titlepage
  \vfill
  \tiny Based on Linux Kernel seccomp-bpf (Kernel Documentation, 2005-2025).
\end{frame}

% ---------------------------
% Slide 2: Problem Statement
% ---------------------------
\begin{frame}{Problem Statement}
\justifying
\begin{itemize}
    \item \textbf{Problem:} Modern applications require sandboxing untrusted code and restricting system call access for security
    \item \textbf{Objectives:} 
    \begin{itemize}
        \item Implement fine-grained system call filtering using seccomp-BPF
        \item Create practical sandbox environments for containers
        \item Prevent malicious syscalls while maintaining performance
    \end{itemize}
    \item \textbf{Approach:} Berkeley Packet Filter (BPF) with Linux kernel seccomp feature
    \item \textbf{Motivation:} Essential for container security (Docker, Chrome, systemd), preventing privilege escalation
    \item \textbf{Applications:} Containerization, browser sandboxing, cloud security, untrusted code execution
\end{itemize}
\end{frame}

% ---------------------------
% Slide 3: Dataset / System Setup
% ---------------------------
\begin{frame}{System Setup \& Development Environment}
\justifying
\begin{itemize}
    \item \textbf{Test Workloads:} Shell commands, file operations, network services
    \item \textbf{System Requirements:}
    \begin{itemize}
        \item Linux Kernel 3.5+ (for seccomp-bpf support)
        \item Docker/Container runtime
        \item GCC compiler, libseccomp-dev
    \end{itemize}
    \item \textbf{Development:} C programming language
    \item \textbf{Tools:} Docker Compose for containerized testing, Makefile build system
    \item \textbf{Testing:} Multiple sandbox levels (strict, limited, network, custom)
\end{itemize}
\end{frame}

% ---------------------------
% Slide 4: Design / Implementation
% ---------------------------
\begin{frame}{Design \& Implementation}
\begin{columns}[T]
\column{0.55\textwidth}
\justifying
\begin{itemize}
    \item \textbf{Architecture:}
    \begin{itemize}
        \item BPF filter rules loaded into kernel
        \item System call interception layer
        \item Policy enforcement engine
    \end{itemize}
    \item \textbf{Implementation:}
    \begin{itemize}
        \item Basic seccomp (strict mode)
        \item Advanced BPF filtering
        \item Sandbox levels: file ops, network, shell
    \end{itemize}
    \item \textbf{Metrics:} Syscall rejection rate, execution overhead, security violations
\end{itemize}

\column{0.45\textwidth}
\centering
\small
\textbf{[PLACEHOLDER: Architecture Diagram]}
\vspace{0.3cm}

Show: Application $\rightarrow$ Seccomp Filter $\rightarrow$ Kernel

Include: BPF rules, Allow/Deny decisions, Syscall flow
\end{columns}
\end{frame}

% ---------------------------
% Slide 5: Results – 1 (Qualitative)
% ---------------------------
\begin{frame}{Results – Qualitative}
\justifying
\begin{itemize}
    \item \textbf{Successful blocking of restricted syscalls}
    \begin{itemize}
        \item Attempted \texttt{execve()} blocked with EACCES
        \item Network calls (\texttt{socket()}, \texttt{connect()}) denied in network sandbox
        \item File operations restricted outside allowed directories
    \end{itemize}
    \item \textbf{Sandbox modes verified:}
    \begin{itemize}
        \item Strict: Only read/write/exit allowed
        \item Limited: Controlled file access
        \item Network: No socket operations
    \end{itemize}
\end{itemize}
\vspace{0.5cm}
\centering
\small
\textbf{[PLACEHOLDER: Terminal Screenshot]}

Show output from \texttt{./bin/sandbox\_example} with blocked syscall attempt
\end{frame}

% ---------------------------
% Slide 6: Results – 2 (Qualitative/Quantitative)
% ---------------------------
\begin{frame}{Results – Performance Comparison}
\centering
\begin{table}[h]
\centering
\begin{tabular}{lll}
\toprule
\textbf{Feature / Test} & \textbf{Result} & \textbf{Conclusion} \\
\midrule
Basic Execution & $\sim$0.01--0.03s & Fast baseline \\
Syscalls w/ Seccomp & $\sim$+1--3ms overhead & Negligible performance impact \\
Read-Only Sandbox & Reads \checkmark{} / Writes $\times$ & Policy enforced correctly \\
Network Sandbox & Network $\times$ / I/O \checkmark{} & Isolation works perfectly \\
Stability & No crashes & Reliable system \\
\bottomrule
\end{tabular}
\caption{Performance and Security Validation Results}
\end{table}
\end{frame}

% ---------------------------
% Slide 7: Results – 3 (Quantitative)
% ---------------------------
\begin{frame}{Results}
\justifying
\begin{itemize}
    \item \textbf{Filter Effectiveness:}
    \begin{itemize}
        \item 100\% syscall blocking accuracy for restricted calls
        \item Zero false negatives (no blocked calls executed)
        \item Minimal performance overhead (<5\%)
    \end{itemize}
    \item \textbf{Sandbox Validation:}
    \begin{itemize}
        \item Successfully prevented privilege escalation attempts
        \item File operations restricted to allowed paths
        \item Network isolation verified
    \end{itemize}
    \item \textbf{Test Coverage:}
    \begin{itemize}
        \item Tested 50+ system calls across different categories
        \item Multiple sandbox levels validated
        \item Edge cases: fork(), clone(), execve() handled correctly
    \end{itemize}
\end{itemize}
\end{frame}

% ---------------------------
% Slide 8: Conclusion
% ---------------------------
\begin{frame}{Conclusion}
\justifying
\textbf{Summary of Contributions:}
\begin{itemize}
    \item \textbf{Devesh Sawant:} Core seccomp filter implementation, BPF architecture design, Docker containerization
    \item \textbf{Aadi Singhal:} Sandbox examples (file ops, network), testing framework, documentation
    \item \textbf{Harshpreet Singh Sodhi:} Security validation, shell sandbox implementation, performance analysis
\end{itemize}

\textbf{Limitations:}
\begin{itemize}
    \item Architecture-specific (x86-64 focus)
    \item Requires kernel 3.5+ for BPF support
    \item Complex filter rules can impact maintainability
    \item Limited to syscall-level filtering (no content inspection)
\end{itemize}

\textbf{Future Work:}
\begin{itemize}
    \item Integration with eBPF for enhanced capabilities
    \item Machine learning-based anomaly detection
    \item Dynamic filter updates without restart
    \item Support for additional architectures (ARM, RISC-V)
\end{itemize}
\end{frame}

% ---------------------------
% Slide 9: References
% ---------------------------
\begin{frame}{References}
\footnotesize
\begin{thebibliography}{99}
\bibitem{kernel} Linux Kernel Documentation: seccomp-bpf. \url{https://www.kernel.org/doc/Documentation/prctl/seccomp_filter.txt}

\bibitem{manpage} seccomp(2) - Linux manual page. \url{https://man7.org/linux/man-pages/man2/seccomp.2.html}

\bibitem{docker} Docker Security Documentation - Seccomp profiles. \url{https://docs.docker.com/engine/security/seccomp/}
\end{thebibliography}
\end{frame}

\end{document}
